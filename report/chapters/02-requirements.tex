\chapter{Requirements Analysis}
\label{ch:requirements}

This chapter presents a comprehensive requirements analysis based on extensive survey of existing Linux process management tools, user interviews, online forum discussions, and academic literature. Requirements are categorized into functional and non-functional specifications with clear priorities and acceptance criteria.

\section{Survey Methodology}

To establish requirements, we conducted a systematic survey of the process management ecosystem:

\subsection{Tool Analysis}

We performed hands-on evaluation of the following tools:

\begin{itemize}[leftmargin=*]
    \item \textbf{Traditional Tools}: \code{ps} (procps-ng 3.3.17), \code{top} (procps-ng 3.3.17), \code{kill}
    \item \textbf{Enhanced Monitors}: \code{htop} 3.2.1, \code{atop} 2.7.1
    \item \textbf{Modern Tools}: \code{glances} 3.4.0, \code{bashtop}, \code{btop}
    \item \textbf{Specialized Tools}: \code{docker stats}, \code{nvidia-smi}, \code{systemd-cgtop}
\end{itemize}

\subsection{User Research}

Gathered requirements from:
\begin{itemize}[leftmargin=*]
    \item System administrator workflows (Reddit r/linuxadmin, r/sysadmin)
    \item Developer pain points (Stack Overflow, GitHub issues)
    \item DevOps use cases (DevOps forums, monitoring discussions)
    \item Academic requirements (Operating Systems textbooks)
\end{itemize}

\subsection{Gap Analysis}

Identified critical missing features in existing tools:
\begin{enumerate}[leftmargin=*]
    \item Container awareness and Kubernetes integration
    \item GPU monitoring across multiple vendors
    \item Per-process network attribution
    \item Historical data for trend analysis
    \item Remote access capabilities (API, Web UI)
    \item Anomaly detection and intelligent alerting
    \item Metrics export to monitoring systems
    \item Process state comparison and diffing
\end{enumerate}

\section{Functional Requirements}

Functional requirements define what the system must do, organized by priority levels.

\subsection{Priority 1: Core Features (Must Have)}

\subsubsection{FR1: Process Display}

\textbf{Description:} Display comprehensive information about all running processes.

\textbf{Acceptance Criteria:}
\begin{itemize}[leftmargin=*]
    \item Show PID, PPID, name, user, CPU\%, memory usage, state
    \item Update display in real-time with configurable refresh rate
    \item Support display of 1,000+ processes without performance degradation
    \item Parse process information from \code{/proc/[pid]/} filesystem
    \item Handle process creation/termination gracefully
\end{itemize}

\textbf{Rationale:} Core functionality present in all surveyed tools; essential baseline.

\subsubsection{FR2: Process Control}

\textbf{Description:} Send signals to processes for control and termination.

\textbf{Acceptance Criteria:}
\begin{itemize}[leftmargin=*]
    \item Support signals: SIGTERM (15), SIGKILL (9), SIGHUP (1), SIGINT (2), SIGSTOP (19), SIGCONT (18), SIGUSR1 (10), SIGUSR2 (12)
    \item Verify process ownership before sending signals
    \item Provide interactive signal selection interface
    \item Display confirmation dialog for destructive operations
    \item Show success/failure feedback with error messages
\end{itemize}

\textbf{Rationale:} Essential for process management; addresses safety concerns identified in survey.

\subsubsection{FR3: Sorting}

\textbf{Description:} Sort processes by any displayed column.

\textbf{Acceptance Criteria:}
\begin{itemize}[leftmargin=*]
    \item Sort by: PID, name, user, CPU\%, memory\%, start time
    \item Toggle ascending/descending order
    \item Perform sorting in O(n log n) time or better
    \item Maintain sort order across refresh cycles
    \item Provide visual indicator of current sort column
\end{itemize}

\textbf{Rationale:} Top user request in htop issue tracker; critical for workflow.

\subsubsection{FR4: Filtering}

\textbf{Description:} Filter processes based on criteria.

\textbf{Acceptance Criteria:}
\begin{itemize}[leftmargin=*]
    \item Filter by username
    \item Filter by process name pattern (regex support)
    \item Filter by resource threshold (CPU\% > X, memory > Y)
    \item Combine multiple filters (AND logic)
    \item Show filtered count in status bar
\end{itemize}

\textbf{Rationale:} Frequently requested; reduces information overload.

\subsubsection{FR5: Tree View}

\textbf{Description:} Display parent-child process relationships hierarchically.

\textbf{Acceptance Criteria:}
\begin{itemize}[leftmargin=*]
    \item Build process tree from PPID relationships
    \item Show visual tree structure with ASCII art
    \item Toggle between flat and tree views
    \item Maintain expand/collapse state
    \item Navigate tree with keyboard (arrows)
\end{itemize}

\textbf{Rationale:} Critical for debugging; available in htop but not basic tools.

\subsubsection{FR6: Real-time Updates}

\textbf{Description:} Automatically refresh process information.

\textbf{Acceptance Criteria:}
\begin{itemize}[leftmargin=*]
    \item Default refresh interval: 1-2 seconds
    \item Configurable range: 0.1 to 60 seconds
    \item Pause/resume updates on demand
    \item Manual refresh on keypress
    \item Maintain scroll position during updates
\end{itemize}

\textbf{Rationale:} Standard feature; user feedback emphasized configuration importance.

\subsection{Priority 2: Advanced Features (Should Have)}

\subsubsection{FR7: Network Monitoring}

\textbf{Description:} Track network connections per process.

\textbf{Acceptance Criteria:}
\begin{itemize}[leftmargin=*]
    \item Count open network connections per process
    \item Parse \code{/proc/[pid]/fd} and socket inodes
    \item Match socket inodes to \code{/proc/net/tcp}, \code{/proc/net/udp}
    \item Display connection count in process list
    \item Support IPv4 and IPv6
\end{itemize}

\textbf{Rationale:} Gap identified in survey; critical for troubleshooting network issues.

\subsubsection{FR8: Container Awareness}

\textbf{Description:} Detect and label containerized processes.

\textbf{Acceptance Criteria:}
\begin{itemize}[leftmargin=*]
    \item Detect Docker containers via cgroup paths
    \item Detect Kubernetes pods and namespaces
    \item Support LXC, Podman, containerd
    \item Show container ID and runtime type
    \item Display resource limits from cgroups
    \item Visual indicator (emoji/symbol) for containers
\end{itemize}

\textbf{Rationale:} Critical gap in traditional tools; essential for cloud-native environments.

\subsubsection{FR9: Historical Data}

\textbf{Description:} Store process metrics over time.

\textbf{Acceptance Criteria:}
\begin{itemize}[leftmargin=*]
    \item Store data in SQLite database
    \item Record CPU\%, memory usage, process count
    \item Configurable recording interval (1-300 seconds)
    \item Query historical data by time range
    \item Implement data retention policy (default 30 days)
    \item Database size limit and automatic cleanup
\end{itemize}

\textbf{Rationale:} Addresses "wish I could see past data" user request; enables trend analysis.

\subsubsection{FR10: System Graphs}

\textbf{Description:} Visualize system-wide resource usage.

\textbf{Acceptance Criteria:}
\begin{itemize}[leftmargin=*]
    \item Display CPU usage sparkline
    \item Display memory usage sparkline
    \item Use ASCII/Unicode characters for graphs
    \item Update graphs in real-time
    \item Toggle graph visibility
\end{itemize}

\textbf{Rationale:} Visual feedback highly valued in user surveys; improves usability.

\subsubsection{FR11: Process Search}

\textbf{Description:} Search for processes using patterns.

\textbf{Acceptance Criteria:}
\begin{itemize}[leftmargin=*]
    \item Support regular expressions
    \item Search process name and command line
    \item Case-insensitive matching
    \item Highlight matching processes
    \item Clear search to show all processes
\end{itemize}

\textbf{Rationale:} htop GitHub issues frequently request improved search.

\subsubsection{FR12: Batch Operations}

\textbf{Description:} Operate on multiple processes simultaneously.

\textbf{Acceptance Criteria:}
\begin{itemize}[leftmargin=*]
    \item Multi-select processes (Space key)
    \item Kill all selected processes
    \item Stop/continue selected processes
    \item Visual indication of selection
    \item Confirmation for batch operations
\end{itemize}

\textbf{Rationale:} Workflow efficiency; requested by system administrators.

\subsection{Priority 3: Innovative Features (Nice to Have)}

\subsubsection{FR13: GPU Monitoring}

\textbf{Description:} Monitor GPU usage per process.

\textbf{Acceptance Criteria:}
\begin{itemize}[leftmargin=*]
    \item Support NVIDIA GPUs (\code{nvidia-smi})
    \item Support AMD GPUs (\code{rocm-smi})
    \item Support Intel GPUs (\code{intel\_gpu\_top})
    \item Show GPU memory usage per process
    \item Graceful degradation if GPU unavailable
    \item Visual indicator for GPU-using processes
\end{itemize}

\textbf{Rationale:} Unique feature; addresses ML/gaming user needs.

\subsubsection{FR14: Web UI}

\textbf{Description:} Provide web-based monitoring interface.

\textbf{Acceptance Criteria:}
\begin{itemize}[leftmargin=*]
    \item Serve HTML/JavaScript interface
    \item Display process list with sorting
    \item Support basic process control (kill)
    \item Auto-refresh with configurable interval
    \item Mobile-responsive design
    \item No external dependencies (self-contained)
\end{itemize}

\textbf{Rationale:} Addresses remote monitoring need; competitive advantage.

\subsubsection{FR15: REST API}

\textbf{Description:} Provide HTTP API for programmatic access.

\textbf{Acceptance Criteria:}
\begin{itemize}[leftmargin=*]
    \item Implement RESTful endpoints
    \item List all processes (GET /api/processes)
    \item Get process details (GET /api/processes/:pid)
    \item Kill process (POST /api/processes/:pid/kill)
    \item System information (GET /api/system)
    \item Historical data (GET /api/history)
    \item JSON response format
    \item CORS support for web clients
\end{itemize}

\textbf{Rationale:} DevOps automation requirement; enables integration.

\subsubsection{FR16: Metrics Export}

\textbf{Description:} Export metrics in standard formats.

\textbf{Acceptance Criteria:}
\begin{itemize}[leftmargin=*]
    \item Prometheus text exposition format
    \item InfluxDB line protocol format
    \item Export to file or stdout
    \item Include process and system metrics
    \item Proper metric naming and labels
\end{itemize}

\textbf{Rationale:} Integration with monitoring stacks; DevOps requirement.

\subsubsection{FR17: Anomaly Detection}

\textbf{Description:} Detect unusual process behavior.

\textbf{Acceptance Criteria:}
\begin{itemize}[leftmargin=*]
    \item Z-score analysis for CPU spikes (> 3 std deviations)
    \item Memory growth detection (> 50\% in 1 minute)
    \item Process count anomalies
    \item Configurable thresholds
    \item Alert generation for detected anomalies
\end{itemize}

\textbf{Rationale:} Proactive monitoring; reduces MTTR for incidents.

\subsubsection{FR18: Kubernetes Integration}

\textbf{Description:} Kubernetes pod-level aggregation.

\textbf{Acceptance Criteria:}
\begin{itemize}[leftmargin=*]
    \item Parse pod name from cgroup paths
    \item Aggregate processes by pod
    \item Show namespace information
    \item Track pod resource limits
    \item Detect sidecar containers
\end{itemize}

\textbf{Rationale:} Cloud-native requirement; unique feature.

\subsection{Additional Features (Phase IV)}

\subsubsection{FR19: Logging System}

\textbf{Acceptance Criteria:}
\begin{itemize}[leftmargin=*]
    \item Structured logging with tracing crate
    \item Multiple log levels (ERROR, WARN, INFO, DEBUG, TRACE)
    \item Log rotation (daily, hourly, size-based)
    \item JSON and plain text formats
    \item Performance metrics logging
\end{itemize}

\subsubsection{FR20: CPU Affinity Management}

\textbf{Acceptance Criteria:}
\begin{itemize}[leftmargin=*]
    \item Get CPU affinity mask per process
    \item Set CPU affinity (pin to cores)
    \item Adjust nice values (-20 to 19)
    \item View scheduling policy
    \item Affinity string parsing ("0-3,7-9")
\end{itemize}

\subsubsection{FR21: Smart Alerts}

\textbf{Acceptance Criteria:}
\begin{itemize}[leftmargin=*]
    \item Email notifications (SMTP)
    \item Webhook notifications (HTTP POST)
    \item Desktop notifications
    \item Configurable alert rules
    \item Cooldown periods to prevent spam
\end{itemize}

\subsubsection{FR22: Process Snapshots}

\textbf{Acceptance Criteria:}
\begin{itemize}[leftmargin=*]
    \item Capture complete system state
    \item Save snapshots to disk
    \item Load and compare snapshots
    \item Export to JSON, CSV, HTML
    \item Metadata (tags, descriptions)
\end{itemize}

\subsubsection{FR23: Memory Maps}

\textbf{Acceptance Criteria:}
\begin{itemize}[leftmargin=*]
    \item Parse /proc/[pid]/maps
    \item Categorize regions (code, data, heap, stack, libraries)
    \item ASCII visualization
    \item Library usage summary
    \item Export to CSV and HTML
\end{itemize}

\section{Non-Functional Requirements}

\subsection{Performance Requirements}

\subsubsection{NFR1: Resource Efficiency}

\begin{itemize}[leftmargin=*]
    \item \textbf{Memory Usage}: < 10MB base, < 50MB with 1,000 processes
    \item \textbf{CPU Usage}: < 2\% during normal operation
    \item \textbf{Startup Time}: < 500ms cold start
    \item \textbf{Refresh Latency}: < 100ms for process scan
\end{itemize}

\textbf{Rationale:} Monitor must not impact system performance.

\subsubsection{NFR2: Scalability}

\begin{itemize}[leftmargin=*]
    \item Support 10,000+ processes without degradation
    \item Handle 30 days of historical data (1-minute intervals)
    \item API supports 100+ concurrent users
    \item Database queries complete in < 50ms
\end{itemize}

\textbf{Rationale:} Enterprise servers may have thousands of processes.

\subsection{Reliability Requirements}

\subsubsection{NFR3: Error Handling}

\begin{itemize}[leftmargin=*]
    \item Graceful handling of missing /proc entries
    \item Recovery from permission denied errors
    \item Proper cleanup on abnormal termination
    \item No panic() in production code (except for unrecoverable errors)
\end{itemize}

\subsubsection{NFR4: Data Integrity}

\begin{itemize}[leftmargin=*]
    \item SQLite transactions for atomic updates
    \item Snapshot files include checksums
    \item Configuration validation on load
    \item Crash recovery for database
\end{itemize}

\subsection{Usability Requirements}

\subsubsection{NFR5: User Interface}

\begin{itemize}[leftmargin=*]
    \item Intuitive keyboard shortcuts similar to htop
    \item Context-sensitive help (F1/h key)
    \item Clear status messages for operations
    \item Color-coding for visual clarity
    \item Minimum terminal size: 80x24 characters
\end{itemize}

\subsubsection{NFR6: Documentation}

\begin{itemize}[leftmargin=*]
    \item Comprehensive README with examples
    \item Inline code documentation (rustdoc)
    \item API documentation (OpenAPI/Swagger)
    \item Man page for CLI usage
\end{itemize}

\subsection{Security Requirements}

\subsubsection{NFR7: Privilege Management}

\begin{itemize}[leftmargin=*]
    \item Run with user privileges (no setuid)
    \item Verify process ownership before signals
    \item Sanitize user input (regex patterns, file paths)
    \item No execution of external commands with user input
\end{itemize}

\subsubsection{NFR8: Memory Safety}

\begin{itemize}[leftmargin=*]
    \item Leverage Rust's ownership system
    \item No buffer overflows or use-after-free
    \item Minimize unsafe code blocks
    \item Audit unsafe code in code review
\end{itemize}

\subsection{Compatibility Requirements}

\subsubsection{NFR9: Platform Support}

\begin{itemize}[leftmargin=*]
    \item Linux kernel 3.10+ (RHEL 7 baseline)
    \item Support x86\_64 and ARM64 architectures
    \item Compatible with glibc and musl
    \item Tested on: Ubuntu 18.04+, Debian 10+, RHEL 7+, Arch Linux
\end{itemize}

\subsubsection{NFR10: Build System}

\begin{itemize}[leftmargin=*]
    \item Cargo for dependency management
    \item Static linking option for portability
    \item Reproducible builds
    \item CI/CD ready (GitHub Actions, GitLab CI)
\end{itemize}

\subsection{Maintainability Requirements}

\subsubsection{NFR11: Code Quality}

\begin{itemize}[leftmargin=*]
    \item Pass \code{cargo clippy} with no warnings
    \item Formatted with \code{cargo fmt}
    \item Modular architecture (< 500 lines per file recommended)
    \item Test coverage > 80\%
\end{itemize}

\subsubsection{NFR12: Testing}

\begin{itemize}[leftmargin=*]
    \item Unit tests for all modules
    \item Integration tests for features
    \item Performance benchmarks
    \item Continuous integration
\end{itemize}

\section{Requirements Traceability Matrix}

Table \ref{tab:traceability} maps requirements to implementation modules and test cases.

\begin{table}[H]
\centering
\caption{Requirements Traceability Matrix (Sample)}
\label{tab:traceability}
\small
\begin{tabularx}{\textwidth}{|l|X|l|l|}
\hline
\textbf{Req ID} & \textbf{Requirement} & \textbf{Module} & \textbf{Tests} \\
\hline
FR1 & Process Display & process.rs, ui.rs & test\_process\_refresh \\
\hline
FR2 & Process Control & main.rs & test\_signal\_handling \\
\hline
FR3 & Sorting & main.rs & test\_sorting \\
\hline
FR8 & Container Awareness & containers.rs & test\_container\_detection \\
\hline
FR13 & GPU Monitoring & gpu.rs & test\_gpu\_detection \\
\hline
FR15 & REST API & api.rs & test\_api\_endpoints \\
\hline
NFR1 & Resource Efficiency & All modules & benchmarks.rs \\
\hline
NFR7 & Privilege Management & main.rs, process.rs & test\_permissions \\
\hline
\end{tabularx}
\end{table}

\section{Requirements Validation}

Requirements were validated through:

\begin{enumerate}[leftmargin=*]
    \item \textbf{Prototype Testing}: Early prototype validated core requirements with users
    \item \textbf{Comparison Analysis}: Feature matrix compared against htop, glances, atop
    \item \textbf{Peer Review}: Requirements reviewed by course instructor and peers
    \item \textbf{Acceptance Testing}: Each requirement has objective acceptance criteria
\end{enumerate}

\section{Summary}

This chapter established comprehensive requirements based on systematic survey and gap analysis. The requirements are organized by priority (must have, should have, nice to have) and category (functional, non-functional). All requirements include clear acceptance criteria and rationale. The next chapter details the architectural design that fulfills these requirements.
