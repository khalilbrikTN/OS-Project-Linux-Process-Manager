\chapter{Screenshots and UI Documentation}
\label{app:screenshots}

This appendix provides detailed visual documentation of the Linux Process Manager's user interfaces, including terminal UI (TUI) screens, web UI views, and various operational modes. Screenshots illustrate key features and workflows described in the main chapters.

\textit{Note: This appendix describes screenshot placeholders. Actual screenshots should be captured from the running application and inserted using \textbackslash includegraphics commands.}

\section{Terminal User Interface (TUI)}
\label{sec:tui-screenshots}

\subsection{Main Process View}

\begin{figure}[h]
\centering
\fbox{\parbox{0.95\textwidth}{
\texttt{Screenshot: figures/tui-main-view.png}

\textbf{Description}: Main TUI interface showing process list with default column layout. The screen displays approximately 20-25 processes with columns for PID, User, CPU\%, MEM\%, Memory (KB), and Name. The selected process is highlighted in blue. System information bar at top shows CPU count, load averages, memory usage, and uptime. Status bar at bottom displays current operation mode and keyboard shortcuts.

\textbf{Key elements visible}:
\begin{itemize}
    \item System info bar with 8 cores, load averages 1.2/1.5/1.8
    \item Memory usage: 45.2\% (7.2G/16G)
    \item Process count: 245 processes
    \item Sorted by CPU usage (descending)
    \item Selected process: PID 1234 (firefox, 45.2\% CPU)
\end{itemize}
}}
\caption{TUI Main Process View}
\label{fig:tui-main}
\end{figure}

\subsection{Tree View Mode}

\begin{figure}[h]
\centering
\fbox{\parbox{0.95\textwidth}{
\texttt{Screenshot: figures/tui-tree-view.png}

\textbf{Description}: Process tree view showing parent-child relationships using Unicode box-drawing characters. Demonstrates hierarchical structure from systemd (PID 1) down through user session processes. Indentation levels clearly indicate process depth. Each process shows PID, user, CPU\%, memory\%, and name.

\textbf{Visible hierarchy example}:
\begin{verbatim}
systemd (1)
├─ NetworkManager (456)
├─ gnome-session (789)
│  ├─ gnome-terminal (123)
│  │  ├─ bash (124)
│  │  │  └─ python3 (125)
\end{verbatim}

\textbf{Features demonstrated}:
\begin{itemize}
    \item Multi-level nesting (up to 5 levels visible)
    \item Process ancestry clearly visible
    \item Daemon processes and their children grouped
\end{itemize}
}}
\caption{TUI Tree View Mode}
\label{fig:tui-tree}
\end{figure}

\subsection{System Resource Graphs}

\begin{figure}[h]
\centering
\fbox{\parbox{0.95\textwidth}{
\texttt{Screenshot: figures/tui-graphs.png}

\textbf{Description}: System resource graphs showing CPU, memory, and swap usage over last 60 seconds using sparkline visualization. Graphs appear above the process list, providing historical context for system load.

\textbf{Graph details}:
\begin{itemize}
    \item CPU: 20-character sparkline showing recent spike to 75\%
    \item Memory: Steady at ~45\% with gradual upward trend
    \item Swap: Minimal usage (<3\%) shown as flat line
    \item Update frequency: 1 second (60 data points visible)
\end{itemize}

\textbf{Visual indicators}:
\begin{itemize}
    \item Characters: ▁▂▃▄▅▆▇█ (8 levels)
    \item Current values displayed at end of each graph
    \item Color coding: green (<50\%), yellow (50-80\%), red (>80\%)
\end{itemize}
}}
\caption{System Resource Graphs}
\label{fig:tui-graphs}
\end{figure}

\subsection{Kill Dialog}

\begin{figure}[h]
\centering
\fbox{\parbox{0.95\textwidth}{
\texttt{Screenshot: figures/tui-kill-dialog.png}

\textbf{Description}: Modal dialog for sending signals to selected process (PID 1234, firefox). Dialog overlays main process view with darkened background. Lists available signals with keyboard shortcuts and descriptions.

\textbf{Dialog content}:
\begin{verbatim}
┌─── Send Signal to Process 1234 (firefox) ───────┐
│                                                  │
│  Select signal to send:                         │
│                                                  │
│  [t] SIGTERM (15) - Graceful termination        │
│  [9] SIGKILL (9)  - Force kill (cannot block)   │
│  [1] SIGHUP (1)   - Hangup (reload config)      │
│  [2] SIGINT (2)   - Interrupt (Ctrl+C)          │
│  [s] SIGSTOP (19) - Suspend process             │
│  [c] SIGCONT (18) - Continue suspended process  │
│                                                  │
│  [Enter] Confirm    [Esc] Cancel                │
└──────────────────────────────────────────────────┘
\end{verbatim}

\textbf{User workflow}: Press 'k' $\rightarrow$ select signal $\rightarrow$ confirm/cancel
}}
\caption{Kill Signal Selection Dialog}
\label{fig:kill-dialog}
\end{figure}

\subsection{Search Mode}

\begin{figure}[h]
\centering
\fbox{\parbox{0.95\textwidth}{
\texttt{Screenshot: figures/tui-search.png}

\textbf{Description}: Search mode activated showing regex pattern input at bottom of screen. Process list filtered to show only matches. Search supports full regex syntax with case-insensitive matching.

\textbf{Example search}:
\begin{itemize}
    \item Pattern: \texttt{/python.*}
    \item Results: 8 processes matched
    \item Matched processes highlighted in search results view
    \item Status bar shows "Search: python.* (8 matches)"
\end{itemize}

\textbf{Supported patterns}:
\begin{itemize}
    \item Simple text: \texttt{/firefox}
    \item Regex: \texttt{/chrom[e|ium]}
    \item Anchored: \texttt{/\textasciicircum docker}
    \item Case-sensitive: \texttt{/(?-i)Python}
\end{itemize}
}}
\caption{Process Search with Regex}
\label{fig:tui-search}
\end{figure}

\subsection{Help Overlay}

\begin{figure}[h]
\centering
\fbox{\parbox{0.95\textwidth}{
\texttt{Screenshot: figures/tui-help.png}

\textbf{Description}: Comprehensive help overlay (activated with 'h' or F1) displaying all keyboard shortcuts organized by category. Semi-transparent overlay allows viewing process list underneath.

\textbf{Help sections}:
\begin{enumerate}
    \item Navigation: Arrow keys, Page Up/Down, Home/End
    \item Sorting: p (PID), n (name), u (user), c (CPU), m (memory)
    \item Actions: k (kill), / (search), t (tree), g (graphs)
    \item General: r/F5 (refresh), h/F1 (help), q (quit)
\end{enumerate}

\textbf{Layout}: Two-column format with keyboard shortcut on left, description on right
}}
\caption{Help Overlay}
\label{fig:help-overlay}
\end{figure}

\subsection{Container-Aware View}

\begin{figure}[h]
\centering
\fbox{\parbox{0.95\textwidth}{
\texttt{Screenshot: figures/tui-containers.png}

\textbf{Description}: Process view showing containerized processes with container indicators (emoji badges). Additional column displays container runtime type (Docker/Podman/K8s).

\textbf{Container indicators}:
\begin{itemize}
    \item Docker containers: [D] prefix or whale emoji
    \item Kubernetes pods: [K8s] prefix or ship wheel emoji
    \item Podman containers: [P] prefix
    \item Container ID shown in separate column (truncated to 12 chars)
\end{itemize}

\textbf{Example entries}:
\begin{verbatim}
PID   User  CPU%  Container        Name
1234  root  12.3  [D] a1b2c3d4e5f6  nginx
5678  root   8.5  [K8s] pod-xyz123 app-server
\end{verbatim}
}}
\caption{Container-Aware Process View}
\label{fig:tui-containers}
\end{figure}

\subsection{GPU Monitoring View}

\begin{figure}[h]
\centering
\fbox{\parbox{0.95\textwidth}{
\texttt{Screenshot: figures/tui-gpu.png}

\textbf{Description}: Process view with GPU memory column showing per-process GPU utilization. GPU information panel at top displays system GPU stats.

\textbf{GPU panel content}:
\begin{verbatim}
GPU 0: NVIDIA GeForce RTX 3080 | 2048 MB / 10240 MB (20%)
       Temp: 65°C | Utilization: 45% | Driver: 470.86
\end{verbatim}

\textbf{Process GPU column}:
\begin{itemize}
    \item Shows GPU memory usage in MB for GPU-using processes
    \item Empty for CPU-only processes
    \item Example: "1024 MB" for process using 1GB GPU RAM
\end{itemize}
}}
\caption{GPU Monitoring View}
\label{fig:tui-gpu}
\end{figure}

\section{Web User Interface}
\label{sec:web-screenshots}

\subsection{Web UI Dashboard}

\begin{figure}[h]
\centering
\fbox{\parbox{0.95\textwidth}{
\texttt{Screenshot: figures/web-dashboard.png}

\textbf{Description}: Browser-based web UI showing process table with real-time updates. Modern responsive design with Bootstrap styling. Top navigation bar provides access to different views (Processes, Containers, GPU, History).

\textbf{Dashboard sections}:
\begin{enumerate}
    \item System Overview Cards: CPU, Memory, Swap, Uptime
    \item Filter Panel: Filter by user, CPU threshold, memory threshold
    \item Process Table: Sortable columns, pagination (25/50/100 per page)
    \item Action Buttons: Kill, Refresh, Export CSV
\end{enumerate}

\textbf{Interactive features}:
\begin{itemize}
    \item Click column headers to sort
    \item Hover over processes for full command line
    \item Right-click process for context menu (kill, details)
    \item Auto-refresh toggle (5/10/30 seconds)
\end{itemize}
}}
\caption{Web UI Dashboard}
\label{fig:web-dashboard}
\end{figure}

\subsection{Process Details Modal}

\begin{figure}[h]
\centering
\fbox{\parbox{0.95\textwidth}{
\texttt{Screenshot: figures/web-process-details.png}

\textbf{Description}: Modal dialog showing detailed information for selected process. Displays extended attributes not visible in main table.

\textbf{Details shown}:
\begin{itemize}
    \item Basic Info: PID, PPID, User, UID/GID
    \item Resources: CPU\%, Memory (KB and \%), Threads
    \item Timing: Start time, Running duration, Priority, Nice
    \item Extended: Network connections, Container ID, GPU memory
    \item Command Line: Full command with arguments
\end{itemize}

\textbf{Actions available}:
\begin{itemize}
    \item Send Signal (dropdown menu)
    \item View History (opens history chart)
    \item View Children (tree view)
    \item Export Details (JSON)
\end{itemize}
}}
\caption{Process Details Modal}
\label{fig:web-details}
\end{figure}

\subsection{Historical Charts View}

\begin{figure}[h]
\centering
\fbox{\parbox{0.95\textwidth}{
\texttt{Screenshot: figures/web-history.png}

\textbf{Description}: Time-series charts showing CPU and memory usage for selected process over configurable time range (1 hour, 6 hours, 24 hours, 7 days).

\textbf{Chart features}:
\begin{itemize}
    \item Dual-axis chart: CPU\% (left), Memory MB (right)
    \item Interactive tooltips showing exact values at cursor position
    \item Zoom and pan controls
    \item Export chart as PNG/SVG
    \item Anomaly markers highlighted in red
\end{itemize}

\textbf{Implementation}: Chart.js library with time-series plugin
}}
\caption{Historical Data Charts}
\label{fig:web-history}
\end{figure}

\section{Command-Line Output Examples}
\label{sec:cli-screenshots}

\subsection{Metrics Export - Prometheus Format}

\begin{figure}[h]
\centering
\fbox{\parbox{0.95\textwidth}{
\texttt{Screenshot: figures/cli-prometheus-export.png}

\textbf{Description}: Terminal output showing Prometheus metrics export format.

\textbf{Sample output}:
\begin{verbatim}
$ ./process-manager --export prometheus

# HELP process_cpu_usage Process CPU usage percentage
# TYPE process_cpu_usage gauge
process_cpu_usage{pid="1234",name="firefox",user="john"} 45.2
process_cpu_usage{pid="5678",name="chrome",user="john"} 23.1

# HELP process_memory_bytes Process memory usage in bytes
# TYPE process_memory_bytes gauge
process_memory_bytes{pid="1234",name="firefox"} 1263820800
process_memory_bytes{pid="5678",name="chrome"} 892743680

# HELP system_cpu_count Number of CPU cores
# TYPE system_cpu_count gauge
system_cpu_count 8

# HELP system_load_average System load average
# TYPE system_load_average gauge
system_load_average{period="1m"} 1.2
system_load_average{period="5m"} 1.5
system_load_average{period="15m"} 1.8
\end{verbatim}
}}
\caption{Prometheus Metrics Export}
\label{fig:prometheus-export}
\end{figure}

\subsection{API Server Startup}

\begin{figure}[h]
\centering
\fbox{\parbox{0.95\textwidth}{
\texttt{Screenshot: figures/cli-api-startup.png}

\textbf{Description}: Terminal showing API server startup messages.

\textbf{Startup output}:
\begin{verbatim}
$ ./process-manager --api --api-port 8080

[2024-11-30T12:00:00Z INFO  process_manager::api]
    Starting API server...
[2024-11-30T12:00:00Z INFO  actix_server::builder]
    Starting 8 workers
[2024-11-30T12:00:00Z INFO  actix_server::server]
    Actix runtime found; starting in Actix runtime
[2024-11-30T12:00:01Z INFO  process_manager::api]
    API server listening on http://0.0.0.0:8080
[2024-11-30T12:00:01Z INFO  process_manager::api]
    Web UI available at: http://localhost:8080
[2024-11-30T12:00:01Z INFO  process_manager::api]
    Press Ctrl+C to shutdown

Available endpoints:
  GET    /api/processes
  GET    /api/processes/:pid
  DELETE /api/processes/:pid
  GET    /api/system
  GET    /api/gpu
  GET    /api/containers
  GET    /api/history/processes/:pid
  GET    /api/metrics
\end{verbatim}
}}
\caption{API Server Startup}
\label{fig:api-startup}
\end{figure}

\subsection{Example API Client Script Output}

\begin{figure}[h]
\centering
\fbox{\parbox{0.95\textwidth}{
\texttt{Screenshot: figures/cli-api-client.png}

\textbf{Description}: Output from Python CPU monitoring script (examples/api\_monitor\_cpu.py).

\textbf{Sample output}:
\begin{verbatim}
$ python3 examples/api_monitor_cpu.py --threshold 50

Process Manager CPU Monitor
Threshold: 50.0%
Polling interval: 5 seconds
Press Ctrl+C to stop

[2024-11-30 12:00:05] Monitoring 245 processes...
[2024-11-30 12:00:05] ALERT: PID 1234 (firefox) -
    CPU: 78.3% [OVER THRESHOLD]
[2024-11-30 12:00:05] ALERT: PID 5678 (chrome) -
    CPU: 65.1% [OVER THRESHOLD]

[2024-11-30 12:00:10] Monitoring 247 processes...
[2024-11-30 12:00:10] ALERT: PID 1234 (firefox) -
    CPU: 82.5% [OVER THRESHOLD]

[2024-11-30 12:00:15] Monitoring 245 processes...
[2024-11-30 12:00:15] No processes over threshold

Total alerts: 3
High CPU processes: firefox (2), chrome (1)
\end{verbatim}
}}
\caption{API Client Script Output}
\label{fig:api-client}
\end{figure}

\section{Error States and Messages}
\label{sec:error-screenshots}

\subsection{Permission Denied}

\begin{figure}[h]
\centering
\fbox{\parbox{0.95\textwidth}{
\texttt{Screenshot: figures/error-permission.png}

\textbf{Description}: Error dialog displayed when attempting to kill process owned by another user without sudo privileges.

\textbf{Error message}:
\begin{verbatim}
┌─── Error ──────────────────────────────────────┐
│                                                │
│  Failed to kill process 1234                  │
│                                                │
│  Permission denied: cannot control process    │
│  owned by another user.                       │
│                                                │
│  Process owner: root (UID: 0)                 │
│  Current user: john (UID: 1000)               │
│                                                │
│  Run with sudo to manage system processes.    │
│                                                │
│  [Press any key to continue]                  │
└────────────────────────────────────────────────┘
\end{verbatim}
}}
\caption{Permission Denied Error}
\label{fig:error-permission}
\end{figure}

\subsection{GPU Not Available}

\begin{figure}[h]
\centering
\fbox{\parbox{0.95\textwidth}{
\texttt{Screenshot: figures/warning-no-gpu.png}

\textbf{Description}: Warning message displayed when GPU monitoring requested but no supported GPU found.

\textbf{Warning banner}:
\begin{verbatim}
┌────────────────────────────────────────────────┐
│ ⚠ GPU Monitoring Unavailable                  │
│                                                │
│ No supported GPU detected or nvidia-smi       │
│ not found in PATH.                            │
│                                                │
│ GPU column will remain empty.                 │
│                                                │
│ Supported GPUs: NVIDIA, AMD, Intel            │
└────────────────────────────────────────────────┘
\end{verbatim}

\textbf{Behavior}: Application continues without GPU data; graceful degradation
}}
\caption{GPU Not Available Warning}
\label{fig:warning-gpu}
\end{figure}

\section{Summary}

This appendix documented the visual interfaces of the Linux Process Manager:

\begin{itemize}
    \item \textbf{TUI Screenshots}: Main view, tree view, graphs, dialogs, search, help, containers, GPU
    \item \textbf{Web UI Screenshots}: Dashboard, process details, historical charts
    \item \textbf{CLI Output}: Metrics export, API server, client scripts
    \item \textbf{Error States}: Permission errors, warnings, graceful degradation
\end{itemize}

Actual screenshots should be captured and inserted at the indicated figure placeholders during final report preparation. Recommended screenshot resolution: 1920x1080, saved as PNG with appropriate compression.

\textbf{Screenshot capture instructions}:
\begin{enumerate}
    \item Launch process manager in various modes
    \item Use \texttt{scrot}, \texttt{gnome-screenshot}, or similar tools
    \item Crop to relevant area (remove desktop chrome)
    \item Save to \texttt{report/figures/} directory
    \item Update \textbackslash includegraphics paths in this file
\end{enumerate}
